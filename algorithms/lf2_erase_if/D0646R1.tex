\documentclass[11pt]{article}

\usepackage{xcolor}
\usepackage{fullpage}
\usepackage[colorlinks, allcolors=blue]{hyperref}
\usepackage{listings}
\usepackage{parskip}

\lstdefinelanguage{diff}{
  morecomment=[f][\color{blue}]{@@},           % group identifier
  morecomment=[f][\color{red}]{-},             % deleted lines
  morecomment=[f][\color{green!50!black}]{+},  % added lines
  morecomment=[f][\color{magenta}]{---},       % diff header lines
  morecomment=[f][\color{magenta}]{+++},
}

\lstset{
  basicstyle=\footnotesize\ttfamily,
}

\setcounter{section}{-1}

\newcommand{\emailaddress}{marc.mutz@kdab.com}
\newcommand{\email}{\href{mailto:\emailaddress}{\emailaddress}}

\date{}
\title{Improving the Return Value of Erase-Like Algorithms I: \texttt{list}/\texttt{forward\_list}}

\begin{document}

\maketitle\vspace{-2cm}

\begin{flushright}
  \begin{tabular}{ll}
% Document \#:&P0646R0\\
  Document \#:&D0646R1\\
  Date:       &\date{2018-06-04}\\
  Project:    &Programming Language C++\\
              &Library Working Group\\
%              &Library Group\\
  Reply-to:   &\author{Marc Mutz} \textless\email\textgreater
  \end{tabular}
\end{flushright}

\section{Change History}

\subsection{Changes from P0646R0}

\begin{enumerate}
\item Removed changes to Library Fundamentals V2, as that is already
  released. Split the Library Fundamentals bits into a new paper to be
  released when LFv3 opens shop.
\item Changed the return type from \texttt{size\_t} to
  \texttt{container::size\_type} (as per LEWG request in Toronto).
\item Rebased on latest C++2a draft \cite{cpp}.
\end{enumerate}

\section{Introduction}

We propose to change the return type of the \texttt{remove()},
\texttt{remove\_if()} and \texttt{unique()} members of
\texttt{forward\_list} and \texttt{list} from \texttt{void} to
\texttt{container::size\_type}, returning the number of elements removed.

This restores consistency with long-established API, such as
\texttt{map/set::erase(key\_type)}.

We show that C++17 compilers do not pessimise existing users that
ignore the return value.

\section{Motivation and Scope}

\subsection{[[nodiscard]] Useful Information}

Alexander Stepanov, in his A9 courses\cite{A9}, teaches us not to
throw away useful information, but instead return it from the
algorithm.

With that in mind, look at the following example:
\begin{lstlisting}[language=C++]
std::forward_list<std::shared_ptr<T>> fl = ...;
fl.remove(nullptr);
\end{lstlisting}
Did \texttt{remove()} remove anything? We don't know. The only way we
\emph{can} learn whether the algorithm removed something is to check
the size of the list before and after the algorithm run. For most
containers, that is a valid option, and fast. All \texttt{size()}
methods of STL containers are $O(1)$ these days.

But \texttt{std::forward\_list} has no \texttt{size()}\ldots

We therefore propose to make the algorithms return the number of
removed elements. While it is only really necessary for
\texttt{forward\_list}, we believe that consistency here is more
important than minimalism.

Returning the number of elements also enables convenient one-line
checks:
\begin{lstlisting}[language=C++]
if (fl.remove(nullptr)) {
    // removed some
}
\end{lstlisting}

\subsection{Consistency}

We note that the associative containers have returned the number of
erased elements from their \texttt{erase(key\_type)} member functions
since at least \cite{STL}. This proposal therefore also restores
lost consistency with existing practice.

\section{Impact on the Standard}

Minimal. We propose to change the return value of library functions
from \texttt{void} to \texttt{container::size\_type}. Existing users
expecting no return value can continue to ignore it. In particular,
this is one of the changes explicitly mentioned in \cite{P0921R2}.

\section{Proposed Wording}

\subsection{Changes to \cite{cpp}}

In section \textbf{[forwardlist.overview]}:

\begin{itemize}
\item in paragraph 3, change the \texttt{remove()},
  \texttt{remove\_if()} and \texttt{unique()} return types from
  \texttt{void} to \texttt{size\_type} (four instances).
\end{itemize}

In section \textbf{[forwardlist.ops]}:

\begin{itemize}
\item after paragraphs 12 and 16, change the \texttt{remove()},
  \texttt{remove\_if()} and \texttt{unique()} return types from
  \texttt{void} to \texttt{size\_type} (four instances).
\item after paragraphs 13 and 17, add new paragraph each:
  \begin{quotation}
    \textit{Returns:} The number of elements erased.
  \end{quotation}
\end{itemize}

In section \textbf{[list.overview]}:

\begin{itemize}
\item in paragraph 2, change the \texttt{remove()},
  \texttt{remove\_if()} and \texttt{unique()} return types from
  \texttt{void} to \texttt{size\_type} (four instances).
\end{itemize}

In section \textbf{[list.ops]}:

\begin{itemize}
\item after paragraphs 14 and 18, change the \texttt{remove()},
  \texttt{remove\_if()} and \texttt{unique()} return types from
  \texttt{void} to \texttt{size\_type} (four instances).
\item after paragraphs 15 and 19, add new paragraph each:
  \begin{quotation}
    \textit{Returns:} The number of elements erased.
  \end{quotation}
\end{itemize}

\section{Performance Considerations}

Early reviewers of this proposal expressed concerns that the
calculation of the return value might pessimise the algorithm over the
version that returns \texttt{void}. Tests run on \url{godbolt.org}
show, however, that the assembler instructions generated for the
functions \texttt{counting()} and \texttt{noncounting()} in the
following test were identical for GCC:

\lstinputlisting[language=C++]{code/comparison.cpp}

Clang sometimes formats the code a little differently (same
instructions, grouped differently), without a clear indication which
of the two is better. In Table~\ref{tab:asm}, this is called
\emph{equivalent}.

\begin{table}
  \centering
  \begin{tabular}[t]{|l||c|c|c|}
    \hline
    Container      & GCC 7.1   & Clang 4.0  & MSVC 2017 \\ \hline\hline
    vector         & identical & identical  & --- \\
    deque          & identical & identical  & --- \\
    list           & identical & equivalent & --- \\
    set            & identical & equivalent & --- \\
    unordered\_set & identical & identical  & --- \\
    map            & identical & equivalent & --- \\
    unordered\_map & identical & identical  & --- \\ \hline
  \end{tabular}
  \caption{Assembler Comparison @ \texttt{-O2} (MSVC does not support constexpr-if)}
  \label{tab:asm}
\end{table}

We think it is safe to say that the introduction of the return type
does not pessimise callers that don't need it.

\section{Acknowledgements}

We thank the reviewers of draft versions of this proposal and the
participants of the associated discussions on
\url{std-proposals@isocpp.org} and LWG in Rapperswil for their input: Sean
Parent, Arthur O'Dwyer, Nicol Bolas, Ville Voutilainen, Casey Carter,
Milian Wolff, Andr\'e Somers. All remaining errors are ours.

The authors would like to thank Titus Winters for championing revision
0 of this paper in Toronto.

\section{References}
\renewcommand{\section}[2]{}%
\begin{thebibliography}{9}
\bibitem[A9]{A9}
  Alexander Stepanov \emph{et al.}\newline
  \emph{Four Algorithmic Journeys / Efficient Programming With Components /
    Programming Conversations}\newline
  \url{https://www.youtube.com/user/A9Videos/playlists?view=1}

\bibitem[SGI STL]{STL}
  Alexander Stepanov \emph{et al.}\newline
  \emph{Associative Container}\newline
  in: \emph{Standard Template Library Programmer's Guide}\newline
  \url{https://www.sgi.com/tech/stl/AssociativeContainer.html}

\bibitem[N4750]{cpp}
  Richard Smith (editor)\newline
  \emph{Working Draft, Standard for Programming Language C++}\newline
  \url{http://www.open-std.org/jtc1/sc22/wg21/docs/papers/2018/n4750.pdf}

\bibitem[P0921R2]{P0921R2} Titus Winters\newline
  \emph{Standard Library Compatibility}\newline
  \url{http://www.open-std.org/jtc1/sc22/wg21/docs/papers/2018/p0921r2.pdf}
\end{thebibliography}

\end{document}

