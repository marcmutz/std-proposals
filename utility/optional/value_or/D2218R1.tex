\documentclass[11pt]{article}

\usepackage{xcolor}
\usepackage{fullpage}
\usepackage[colorlinks, allcolors=blue]{hyperref}
\usepackage{listings}
\usepackage{parskip}
\usepackage{amssymb}

\lstdefinelanguage{diff}{
  morecomment=[f][\color{blue}]{@@},           % group identifier
  morecomment=[f][\color{red}]{-},             % deleted lines
  morecomment=[f][\color{green!50!black}]{+},  % added lines
  morecomment=[f][\color{magenta}]{---},       % diff header lines
  morecomment=[f][\color{magenta}]{+++},
}

\lstset{
  basicstyle=\footnotesize\ttfamily,
}

\newcommand{\emailaddress}{marc.mutz@hotmail.com}
\newcommand{\email}{\href{mailto:\emailaddress}{\emailaddress}}

\newcommand{\wgpaper}[1]{\href{https://wg21.link/#1}{#1}}
\newcommand{\isref}[1]{\textbf{[\wgpaper{#1}]}}
\newcommand{\isnref}[2]{\textbf{[\href{https://wg21.link/#1\##2}{#1}]/#2}}

\date{}
\title{More flexible \texttt{value\_or()}}
\author{Marc Mutz}

\begin{document}

\maketitle\vspace{-1cm}

\begin{tabular}{ll}
  Document \#:&D2218R1\\
  Date:       &\today\\
  Audience:   &LEWGI\\
  %              &Library Group\\
  Target:     & C++26\\
  Reply-to:   &\author{Marc Mutz} \textless\email\textgreater
\end{tabular}
\begin{abstract}
  We propose to extend the \texttt{value\_or()} member function
  templates in \texttt{optional} and \texttt{expected} by adding a
  default template argument to make requesting default-constructed
  values simpler:
    \begin{lstlisting}[language=c++]
      // now                         // proposed:
      opt.value_or(Type{});          opt.value_or({});
    \end{lstlisting}
    This brings \texttt{value\_or()} in line with other functions
    (most prominently \texttt{exchange()}).
\end{abstract}

\tableofcontents

\setcounter{section}{-1}
\section{Change History}

\subsection{R0 → R1}
\begin{itemize}
\item Dropped \texttt{value\_or\_construct()} and
  \texttt{value\_or\_else()}. The former is too ambitious, the latter
  superseded by C++23's adoption of \texttt{or\_else()}.
\item Extended to \texttt{expected::value\_or()}, added to the IS
  after R0 was published.
\item Rebased onto \cite{cpp2b}.
\end{itemize}

\section{Motivation and Scope}

When using \texttt{optional::value\_or()}, more often than not, the
fall-back value passed is some form of default-constructed value:

\begin{lstlisting}[language=c++]
  optional<int> oi = ~~~;       // (1)
  use(oi.value_or(0));
  optional<bool> ob = ~~~;      // (2)
  use(ob.value_or(false));
  optional<string> os = ~~~;    // (3)
  use(os.value_or(nullptr));    // (a)
  use(os.value_or(""));         // (b)
  use(os.value_or({}));         // (c)
  use(os.value_or(string{});    // (d)
  optional<vector<string>> ov = ~~~;
  use(ov.value_or(~~~???~~~));  // (4)
\end{lstlisting}

While this works fine in case of built-in types (1, 2), it already
fails to be convenient when the payload type is a user-defined type
without literals.

\subsection{How the C++ Developer Became a Gardener}

Here's the tale of a C++ developer trying to use \texttt{value\_or()}
in the \texttt{string} case (3): The developer first tries to use
\texttt{nullptr} (a), which crashes on him at runtime due to
\isnref{char.traits.require}1 in conjunction with
\isnref{string.cons}{13}. The next try (b) succeeds, but may invoke an
unnecessary ``\texttt{strlen}'', so he's told in review to use the
\texttt{string} default constructor instead. So the developer tries
(c) which fails to compile because \texttt{\{\}} fails to deduce the
template argument of \texttt{value\_or()}, which is not defaulted, as
e.g.\ the second argument of \texttt{exchange()} is. Grumpily, the
developer caves in and repeats the type name of the
\texttt{optional}'s \texttt{value\_type} (d).

The next day, he's asked to use a \texttt{optional<vector<string>>}
(4) and decides to quit and become a gardener instead.

\subsection{Defaulting \texttt{value\_or()}'s template argument}
\label{sec:defaulting}

With this change, we'd like to ensure that \texttt{value\_or(\{\})}
works, like \texttt{exchange(\textit{var}, \{\})} does.

We can't just default like this:

\begin{lstlisting}[language=c++]
  template <typename T>
  class optional {
  public:
      ~~~~
      template <typename U = T>
      T value_or(U&&) const;
  };
\end{lstlisting}

as that would prevent moving the argument into the return value when
\texttt{T} is cv-qualified (as in \texttt{optional<const string>}). It
follows that we need to remove cv-qualifiers. This is unaffected by
the proposed resolution to \cite{LWG3424}, btw., see
Section~\ref{sec:relation}. We don't need to remove references, as
\texttt{optional<T\&>} and \texttt{expected<T\&,E>} are ill-formed. If
and when \texttt{optional} and/or \texttt{expected} start to support
references, this needs to be rethought.

\begin{lstlisting}[language=c++]
  template <typename T>
  class optional {
  public:
      ~~~~
      template <typename U = remove_cv_t<T>>
      T value_or(U&&) const;
  };
\end{lstlisting}

This enables developers to write \texttt{value\_or(\{\})}, which is
self-explanatory, as long as you know \texttt{value\_or()} as
currently specified.

It also enables all other braced initializers, not just \texttt{\{\}},
to be passed to \texttt{value\_or()}.

\section{Design Decisions}

If all we wanted was to make it easier to return a default-constructed
\texttt{T}, we could just add a new function
\texttt{value\_or\_default\_initialized()}. This is not proposed,
because it does not address the consistency concern with
\texttt{exchange()}.

See \wgpaper{P2218R0} for rationale on not making \texttt{value\_or()}
variadic instead.

\section{Impact on the Standard}

Only positive. Expressions enabled by this proposal make the use of
\texttt{expected::} or \texttt{optional::} \texttt{value\_or()} easier
and more consistent with the rest of the standard library,
e.g.~\texttt{std::exchange()}. At the same time, no existing code is
broken, because the status quo cannot accept braced intializers as
\texttt{value\_or()} arguments.

\section{Relation to Other Proposals}%
\label{sec:relation}

\begin{itemize}
\item\cite{P2248} is a wider-scope version of this paper, addressing a
  similar issue across \texttt{<algorithm>}.
\item\cite{LWG3424} addresses the same functions, but deals with the
  return value instead of the argument. As it's orthogonal to this
  proposal; neither subsumes the other. We note that the proposed
  wording at the time of writing omits \texttt{exchange::value\_or()},
  which, however, seems to have the same issue as
  \texttt{optional::value\_or()}.
\end{itemize}

\section{Proposed Wording}

All wording is relative to \cite{cpp2b}:

\textbf{Conflict resolution note:} If the proposed resolution to
\cite{LWG3424} is accepted, the intent of this paper is to change
the template-initializer-list, and adopt \cite{LWG3424}'s return
value, see Section~\ref{sec:relation}.

\begin{itemize}
\item In \isref{version.syn}, add a new row
  \begin{lstlisting}[language=c++]
    #define __cpp_lib_value_or      YYYYMML // also in <expected>, <optional>
  \end{lstlisting}
\item Change \isref{optional.optional.general} as indicated:
  \begin{lstlisting}[language=diff]
    constexpr const T&& value() const&&;
-   template<class U> constexpr T value_or(U&&) const&;
-   template<class U> constexpr T value_or(U&&) &&;
+   template<class U=remove_cv_t<T>> constexpr T value_or(U&&) const&;
+   template<class U=remove_cv_t<T>> constexpr T value_or(U&&) &&;

   // [optional.monadic], monadic operations
  \end{lstlisting}
\item Apply the above \texttt{remove\_cv\_t<T>} default argument also
  to the declarations of \texttt{value\_or()} just above
  \isnref{optional.observe}{15} and \isnref{optional.observe}{17}.
\item Change \isref{expected.object.general} as indicated:
  \begin{lstlisting}[language=diff]
    constexpr E&& error() && noexcept;
-   template<class U> constexpr T value_or(U&&) const &;
-   template<class U> constexpr T value_or(U&&) &&;
+   template<class U=remove_cv_t<T>> constexpr T value_or(U&&) const &;
+   template<class U=remove_cv_t<T>> constexpr T value_or(U&&) &&;
    template<class G = E> constexpr E error_or(G&&) const &;
  \end{lstlisting}
  [Note: \isref{expected.void} does not offer \texttt{value\_or()}, so
    needs no changes]
\item Apply the above \texttt{remove\_cv\_t<T>} default argument also
  to the declarations of \texttt{value\_or()} just above
  \isnref{expected.object.obs}{16} and \isnref{expected.object.obs}{18}.
\end{itemize}

\section{Acknowledgements}

The author would like to thank all participants of the LEWG(I)
reflector discussion that led to this proposal, esp.\ Andrzej
Krzemienski for confirming that \texttt{optional::value\_or()}'s
non-defaulted template parameter was not a conscious omission, and
Jonathan Wakely and Tomasz Kamiński for dragging the paper out of its
two-year hiatus.

\section{References}
\renewcommand{\section}[2]{}%
\begin{thebibliography}{9}
\bibitem[LWG3424]{LWG3424}
  Casey Carter (reporter)\newline
  \emph{optional::value\_or should never return a cv-qualified type}\newline
  \url{https://wg21.link/LWG3424}
\bibitem[N4928]{cpp2b}
  Thomas Köppe (editor)\newline
  \emph{Working Draft: Standard for Programming Language C++}\newline
  \url{https://wg21.link/N4928}
\bibitem[P2248R7]{P2248}
  Giuseppe D'Angelo\newline
  \emph{Enabling list-initialization for algorithms}\newline
  \url{https://wg21.link/P2248R7}
\end{thebibliography}

\end{document}

