\documentclass[11pt]{article}

\usepackage{xcolor}
\usepackage{fullpage}
\usepackage[colorlinks, allcolors=blue]{hyperref}
\usepackage{listings}
\usepackage{parskip}
\usepackage{amssymb}
\usepackage[normalem]{ulem}

\lstdefinelanguage{diff}{
  morecomment=[f][\color{blue}]{@@},           % group identifier
  morecomment=[f][\color{red}]{-},             % deleted lines
  morecomment=[f][\color{green!50!black}]{+},  % added lines
  morecomment=[f][\color{magenta}]{---},       % diff header lines
  morecomment=[f][\color{magenta}]{+++},
}

\lstset{
  basicstyle=\footnotesize\ttfamily,
}

\newcommand{\emailaddress}{marc.mutz@hotmail.com}
\newcommand{\email}{\href{mailto:\emailaddress}{\emailaddress}}

\newcommand{\wgpaper}[1]{\href{https://wg21.link/#1}{#1}}
\newcommand{\isref}[1]{\textbf{[\wgpaper{#1}]}}
\newcommand{\isnref}[2]{\textbf{[\href{https://wg21.link/#1\##2}{#1}]/#2}}

\newcommand{\intmaxt}{\texttt{intmax\_t}}
\newcommand{\quecto}{\texttt{quecto}}
\newcommand{\ronto}{\texttt{ronto}}
\newcommand{\yocto}{\texttt{yocto}}
\newcommand{\zepto}{\texttt{zepto}}
\newcommand{\zetta}{\texttt{zetta}}
\newcommand{\yotta}{\texttt{yotta}}
\newcommand{\ronna}{\texttt{ronna}}
\newcommand{\quetta}{\texttt{quetta}}

\date{}
\title{Adding the new SI prefixes}

\begin{document}

\maketitle\vspace{-2cm}

\begin{tabular}{ll}
  Document \#:&P2734R0\\
  Date:       &\today\\
  Project:    &Programming Language C++\\
%              &Library Evolution Group\\
              &Library Group\\
  Reply-to:   &\author{Marc Mutz} \textless\email\textgreater
\end{tabular}
\vspace{1cm}
\begin{abstract}
  We propose to add the missing SI prefixes \quecto{} ($10^{-30}$),
  \ronto{} ($10^{-27}$), as well as \ronna{} ($10^{27}$) and \quetta{}
  ($10^{30}$) to the \texttt{<ratio>} header.
\end{abstract}

\section{Motivation and Scope}

The General Conference on Weights and Measures (CGPM), at its
27th~meeting in November~2022, decided \cite{CGPM-22-3}

\begin{quote}
  \ldots to add to the list of SI prefixes to be used for multiples
  and submultiples of units the following prefixes:

  \begin{tabular}{lll}
    Multiplying factor & Name & Symbol\\\hline\\
    $10^{27}$ & \ronna & R\\
    $10^{-27}$ & \ronto & r\\
    $10^{30}$ & \quetta & Q\\
    $10^{-30}$ & \quecto & q\\
  \end{tabular}
\end{quote}

This decision directly affects \isref{ratio.syn} and \isref{ratio.si},
which contain \texttt{ratio} typedefs for \emph{each} SI prefix. If
the list of SI prefixes grows (it last did in 1991), the corresponding
list of \texttt{ratio} typedefs needs to follow suit.

\section{Implementability \& Impact on the Standard}

The multiplying factors denominated by the new SI prefixes cannot be
represented as a \texttt{ratio} when \intmaxt{} is 64-bit. It is a
property they share with the existing prefixes \yocto, \zepto,
\zetta{} and \yotta, which are therefore optional in the current
IS.

Platforms that can represent $10^{24}$ (\yotta) in \intmaxt{} can also
likely represent $10^{30}$ (\quetta) in it (if \intmaxt{} is 128-bit
($10^{30}<2^{100}$)). Even if \intmaxt{} is an 80- or 96-bit entity,
the problem, from a wording point of view, remains the same:
Non-representable typedefs are not required to be provided.

In particular, this proposal is \emph{orthogonal} to
potential\footnote{At least, this author is not aware of any such
  proposals at the time of writing} proposals that wish to guarantee
availability of \yotta{} etc on all platforms, e.g.\ by replacing the
use of \intmaxt{} in the \texttt{ratio} interface with a larger
extended integer type, an option that may or may not become available
were \cite{gb80} to be resolved (at the time of writing, it isn't).

Such proposals, however, would be much more demanding on committee
time and require LEWG involvement, so this proposal steers clear of
such desires and stays within the existing wording for \yotta{} etc to
deliver the missing SI prefixes with as little effort as possible.

\section{Proposed Wording}

The following is relative to \cite{cpp23}:

\begin{itemize}
\item In \isref{version.syn}, add a row
  \begin{lstlisting}[language=C++]
    #define __cpp_lib_ratio             YYYYMML // also in <ratio>
  \end{lstlisting}
\item Change \isref{ratio.syn} as indicated:
\begin{lstlisting}[language=diff]
   // [ratio.si], convenience SI typedefs
+  using quecto = ratio<1, 1'000'000'000'000'000'000'000'000'000'000>;  // see below
+  using ronto  = ratio<1,     1'000'000'000'000'000'000'000'000'000>;  // see below
   using yocto  = ratio<1,         1'000'000'000'000'000'000'000'000>;  // see below
   using zepto  = ratio<1,             1'000'000'000'000'000'000'000>;  // see below
   [...]
   using zetta  = ratio<            1'000'000'000'000'000'000'000, 1>;  // see below
   using yotta  = ratio<        1'000'000'000'000'000'000'000'000, 1>;  // see below
+  using ronna  = ratio<    1'000'000'000'000'000'000'000'000'000, 1>;  // see below
+  using quetta = ratio<1'000'000'000'000'000'000'000'000'000'000, 1>;  // see below
\end{lstlisting}
Editorial note: re-indent the whole block to preserve the \textgreater--shaped form.
\item Change \isref{ratio.si} as indicated:
  \begin{quote}
    For each of the \textit{typedef-name}s \underline{\quecto, \ronto,
    } \yocto, \zepto, \zetta, \sout{and }\yotta, \underline{\ronna,
      and \quetta, }if both of the constants used in its specification
    are representable by \intmaxt, the typedef is defined; if either
    of the constants is not representable by \intmaxt, the typedef is
    not defined.
  \end{quote}
\end{itemize}

\section{References}
\renewcommand{\section}[2]{}%
\begin{thebibliography}{9}

  \bibitem[CGPM~2022~Resolution~3]{CGPM-22-3}
    \emph{CGPM/2022-Resolutions (FR-EN)}, Resolution~3,\newline
    \url{https://www.bipm.org/documents/20126/64811223/Resolutions-2022.pdf/281f3160-fc56-3e63-dbf7-77b76500990f}

  \bibitem[N4917]{cpp23}
    Thomas Köppe,
    \emph{Working Draft, Standard for Programming Language C++}\newline
    \url{https://wg21.link/N4917}
  \bibitem[LWG3828]{gb80}
    GB NB,
    \emph{Sync intmax\_t and uintmax\_t with C2x}\newline
    \url{https://wg21.link/lwg3828}

\end{thebibliography}

\end{document}

